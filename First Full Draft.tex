\documentclass[twocolumn,11pt]{article}
\setlength{\textheight}{9truein}
\setlength{\topmargin}{-0.9truein}
\setlength{\parindent}{0pt}
\setlength{\parskip}{10pt}
\setlength{\columnsep}{.4in}
%
\newcommand{\beq}{\begin{equation}}
\newcommand{\eeq}{\end{equation}}
%
\usepackage{graphicx}
\usepackage{epsfig}
\usepackage{url}
\usepackage{cuted}
\usepackage{appendix}
\usepackage{bm}
%
\title{Assessing the Astrophysical Gravitational Wave Foreground \\ \vspace{.1in} \large Comparing the Efficacies of Signal Deletion and Signal Subtraction with Respect to Gravitational Wave Telescope Parameters}
\author{Thomas Kilmer}
\date{18 Nov 2016}
%
%
\begin{document}
 \twocolumn[
   \begin{@twocolumnfalse}
   \maketitle
   \setlength{\parindent}{0pt}
   \begin{abstract} 

Gravitational waves from astrophysical sources have been detected and the next generation of gravitational wave telescopes might be able to detect gravitational waves from cosmic background sources. However we expect that these cosmic background signals will be obscured by a substantial astrophysical background which will make analyzing the cosmic background prohibitively difficult. In this paper we assess techniques for removing resolved astrophysical sources from gravitational wave data for the purpose of reducing the astrophysical background and easing analysis of the cosmic gravitational wave background. 
   
  \vspace{.3in} 
     \end{abstract}
    \end{@twocolumnfalse}]

\section*{Introduction}

Advanced LIGO recently detected gravitational waves from a binary black hole merger~\cite{1602.03837}. While estimates for the number of coalescing binary black holes in the universe are imprecise, there are many such binaries~\cite{0912.1074}. This large number of binary black holes produces overlapping, incoherent signals which form a ``background'' of gravitational waves~\cite{1604.02513}. This background is statistically homogeneous and isotropic, despite the discrete nature of the sources which make it up~\cite{1604.02513}. With current advances in gravitational wave detection this background will be measured within the next few years~\cite{1604.02513}.

!!! Insert description of background here. !!!

Binary black holes are not the only thing which contribute to the gravitational wave background. A number of other astrophysical sources contribute to the background including rotating neutron stars~\cite{1206.1330}, neutron star phase transitions~\cite{Marranghello}, instabilities in young neutron stars~\cite{1106.2736}, and core collapse supernovae~\cite{astro-ph/0412277}. There are also cosmological sources which contribute to the gravitational wave background including pre-Big Bang cosmological factors~\cite{hep-th/9211021}~\cite{hep-th/9701146}~\cite{1006.0217}, the amplification of weak gravitational waves (including vacuum fluctuations) during inflation~\cite{Grishchuk1}~\cite{Grishchuk2}, phase transitions in the early universe~\cite{0711.2593}~\cite{0909.0622}~\cite{0901.1661}, and cosmic strings~\cite{hep-th/0410222}. The analysis of gravitational waves produced by the aforementioned cosmological sources could provide insight into the early processes of the universe.

However the analysis of the gravitational wave background produced by cosmological sources is complicated by the inability to separate it from the gravitational wave background produced by astrophysical sources~\cite{1604.02513}. However the astrophysical background can be separated into two parts, an unresolvable astrophysical background in which every source generates signals with signal to noise ratios less than 1 and a resolvable astrophysical background in which every source generates signals with SNRs greater than or equal to 1 called the astrophysical foreground~\cite{1106.5795}. The signals generated by the astrophysical foreground can then be identified and removed from the data, reducing the total astrophysical background and making it easier to identify any cosmic background which may exist. 

This project is not concerned with how the astrophysical foreground signals are identified, which has been discussed in great detail elsewhere~\cite{1602.03839}. The goal of this project is instead to develop tools for removing the signals generated by the astrophysical foreground. Before doing so however we need to develop a metric for determining the efficacy of such tools, which is what we have been doing during Capstone I. 



\section*{Astrophysical Background Signal Density}

Let us begin with the signal density for astrophysical gravitational waves described in~\cite{1112.1898}, shown here in Equation~\ref{eq:signal_density}. $\Omega_{gw}$ is the ratio of gravitational wave energy density to a reference energy density, which is the energy required to close the universe \cite{1602.03847}. Note that this formula only accounts for the gravitational waves produced by binary black hole mergers, though it is still a reasonable model for the total astrophysical background as black hole binary mergers are expected to constitute the vast majority of all astrophysical gravitational wave sources~\cite{1206.1330}. 

{\tiny
\beq
\Omega_{g \omega} = \frac{8 \alpha (\pi G M_c)^{5/3}}{9 H_0^3 c^2} f^{2/3} \int_{0}^{z_{max}} \frac{R_V(z)~dz}{(1+z)^{1/3} E[\Omega_M, \Omega_{\Lambda}, z]}
\label{eq:signal_density} 
\eeq
}{\footnotesize
\begin{eqnarray}
R_v(z) &:& \textrm{Rate~of~Binary~Coalescence} \nonumber \\
E[\Omega_M, \Omega_{\Lambda}, z] &:& \textrm{z-Dependence~of~Comoving~Volume}  \nonumber  \\
(1+z)^{1/3} &:& \textrm{Comoving~Volume~Term} \nonumber \\
z &:& \textrm{Source Redshift} \nonumber \\
z_{max} &:& \textrm{Most Distant Coalescing Binaries} \nonumber \\
\frac{8 (\pi G)^{5/3}}{9 H_0^3 c^2} &:& \textrm{Constant~Factors} \nonumber \\
\alpha &:& \textrm{Constant Scaling Factor} \nonumber \\
M_c &:& \textrm{Binary Chirp Mass} \nonumber \\
f &:& \textrm{Signal Frequency} \nonumber \\
\nonumber
\end{eqnarray}
}

To simplify how we model $\Omega_{gw}$ we can split it into three parts, its constant terms, frequency related terms, and redshift related terms. Keep in mind that the frequency of an orbit, an thus the frequency of a gravitational wave signal, is dependent on the masses of the orbiting objects and so chirp mass is a frequency related term. 

{\footnotesize
\begin{eqnarray}
\Omega_{gw} &=& C~Z(z_{max})~F(f)
\label{eq:parts}\\
C &=&  \frac{8 \lambda (\pi G)^{5/3}}{9 H_0^3 c^2} 
\label{eq:constants}\\
Z(z_{max}) &=& \int_{0}^{z_{max}}\frac{R_V(z)~dz}{(1+z)^{1/3} E[\Omega_M, \Omega_{\Lambda}, z]} 
\label{eq:redshift} \\
F(f) &=& M_c^{5/3} f^{2/3} 
\label{eq:frequency}
\end{eqnarray}}



\section*{Resolving the Astrophysical Foreground}

The success of a technique used to resolve sources can be quantified by what fraction of the background it can resolve, how much of the background it can convert into foreground. That is, a good technique resolves a high fraction of the gravitational wave energy incident on a detector. 

We have the signal density for gravitational waves incident on a detector defined above in Equation~\ref{eq:signal_density}, but this does not give us the quantity of gravitational wave energy incident on a detector and it certainly does not tell us what fraction of that energy is resolvable.

The first thing we need to do to is find the probability distribution functions of signal density with respect to all relevant source parameters. We will be evaluating our PDFs with respect to frequency and redshift parameters, with exact methods described in the below subsections. These PDFs can then be integrated over their relevant parameters to create cumulative distribution functions which describe the total incident signal, as an amplitude, generated by sources within a given section of parameter space. Taking the value of these CDFs for the whole of parameter space returns that incident signal generated by all sources, describing the total gravitational wave energy incident on a detector. 

The reason it's important to generate these PDFs is because signal resolvability is heavily dependent on source parameters and therefore finding what fraction of signal is resolvable is not so simple as multiplying total signal by a resolvability parameter. The fraction of signal that is resolvable varies as the source parameters change, so to find the total resolvable signal we first need to generate PDFs of the resolvable signal with respect to the source parameters. From there we can generate CDFs of the resolvable signal which, when evaluated for the whole of parameter space return the resolvable signal generated by all sources. 

Which means that to find the resolved fraction of gravitational wave energy incident on a detector we need to solve Equation~\ref{eq:lambda}, where $\lambda$ is the resolved fraction of the background. 

\begin{eqnarray}
\lambda &=& \frac{CDF_{tot}(\Omega_{gw,r})}{CDF_{tot}(\Omega_{gw})} 
\label{eq:lambda} \\
\Omega_{gw,r} &:& \textrm{Resolvable Signal} \nonumber \\
\Omega_{gw} &:& \textrm{Total Signal} \nonumber \\
CDF &:& \textrm{Cumulative Distribution Function} \nonumber \\
\nonumber
\end{eqnarray}

This can then be broken down into component parts, exactly as in Equation~\ref{eq:parts}, which we show in Equation~\ref{eq:lambda_parts}. (Note that C, Equation~\ref{eq:constants}, cancels out in the numerator and the denominator of $\lambda$).

{\footnotesize
\begin{eqnarray}
\lambda &=& \lambda_Z~\lambda_F 
\label{eq:lambda_parts} \\
\lambda_Z &=& \frac{CDF_{tot}[Z_r]}{CDF_{tot}[Z]} \nonumber \\
\lambda_F &=& \frac{CDF_{tot}[F_r]}{CDF_{tot}[F]} \nonumber \\
Z_r &:& \textrm{Resolvable Portion of $Z$} \nonumber \\
F_r &:& \textrm{Resolvable Portion of $F$} \nonumber \\
\nonumber
\end{eqnarray}}

In the following subsections we evaluate $CDF_{tot}[Z]$, $CDF_{tot}[Z_r]$, $CDF_{tot}[F]$, and $CDF_{tot}[F_r]$ to find $\lambda_Z$ and $\lambda_F$.



\subsection*{$\bm{Z(z_{max})}$}

Let us consider $Z(z_{max})$ (Equation~\ref{eq:redshift}) and its component parts. $R_V(z)$ is determined by observations of gravitational wave sources and is thus not well defined, given the small number of sources so far observed. For the sake of this paper we will be using the theoretical formula for $R_V(z)$ described in~\cite{1602.04531}. $E[\Omega_M, \Omega_{\Lambda}, z]$ and $z_{max}$ are defined in~\cite{1112.1898}. 

\begin{eqnarray}
R_V(z) &=&  0.015~\frac{(1+z)^{2.7}}{1 + [\frac{1+z}{2.9}]^{5.6}} \frac{M_{\odot}}{Mpc^3yr} \nonumber \\
E[\Omega_M, \Omega_{\Lambda}, z] &=& \sqrt{\Omega_M (1+z^3) + \Omega_{\Lambda}} \nonumber \\
z_{max} &=& 6 \nonumber 
\end{eqnarray}

Using standard $\Lambda$CDM cosmology's values of $\Omega_M$~=~0.3 and $\Omega_{\Lambda}$~=~0.7 this allows us to describe Equation~\ref{eq:redshift} in detail, shown and evaluated in Equation~\ref{eq:redshift_numbers}. 

{\footnotesize
\begin{eqnarray}
Z(z_{max}) &=& \int_{0}^{6}\frac{0.015~(1+z)^{2.37}~dz}{\sqrt{0.7+0.3(1+z)^3}~(1+\frac{(1+z)^{5.6}}{2.9})}
\nonumber \\
Z(z_{max}) &=& 0.100218 \nonumber \\
\label{eq:redshift_numbers}
\end{eqnarray}}

This in turn allows us to determine the probability distribution function of $Z(z_{max})$. The PDF of a definite integral is described as the integrand divided by the integral and we now know the value of that integral. $PDF(Z(z_{max})$ is evaluated in Equation~\ref{eq:PDFz} and depicted in Figure~\ref{fig:PDFz}. This describes the distribution of black hole binaries by their redshifts. 

{\tiny
\beq
PDF[Z](z) = \frac{0.1497~(1+z)^{2.37}}{\sqrt{0.7+0.3(1+z)^3}~(1+\frac{(1+z)^{5.6}}{2.9})}
\label{eq:PDFz}
\eeq}

\begin{figure}[!h!t]
	\centering
	\includegraphics[width=3in]{PDFz.pdf}
	\caption{$PDF[Z](z)$ - Equation~\ref{eq:PDFz}}
	\label{fig:PDFz}
\end{figure} 

From here it is simple to find the CDF of $Z(z_{max})$. The CDF of a PDF is simply the PDF integrated from the PDF's minimum value to an arbitrary value, in this case integrating from 0 to $z$. $CDF[Z]$ is evaluated in Equation~\ref{eq:CDFz} and depicted in Figure~\ref{fig:CDFz}. This describes the fraction of black hole binaries one would observe within a given redshift. 

{\tiny
\beq
CDF[Z](z) = \int_{0}^{z}\frac{0.1497~(1+z)^{2.37}~dz}{\sqrt{0.7+0.3(1+z)^3}~(1+\frac{(1+z)^{5.6}}{2.9})}
\label{eq:CDFz}
\eeq}

\begin{figure}[!h!t]
	\centering
	\includegraphics[width=3in]{CDFz.pdf}
	\caption{$CDF[Z](z)$ - Equation~\ref{eq:CDFz}}
	\label{fig:CDFz}
\end{figure} 

Note how $CDF[Z](z)$ noticeably plateaus as $z$ approaches 6. This indicates that the equation for $R_v(z)$ described in~\cite{1602.04531} does indeed describe a distribution of black hole binaries that tapers off and vanishes as one approaches $z$~=6, which squares with the $z_{max}$ value we obtained in~\cite{1112.1898}.



\subsection*{$\bm{Z_r(z_{max})}$}

Now that we have a numerical description of $CDF[Z]$ we can work on finding a numerical description of $CDF[Z_r]$. The first thing we need to do to make that happen is find an equation that describes the resolvability of sources with respect to redshift. For that we turn to \cite{1405.7016}, which describes in detail Equation~\ref{eq:Pok}. (Unfortunately $P_{ok}$ is too complicated to be fully described here). 

{\footnotesize
\begin{eqnarray}
P_{ok}(\frac{D_L(z)}{D_h(M_c)}) &:& \textrm{Cumulative Angle Factor}
\label{eq:Pok} \\
D_L(z) &:& \textrm{Luminosity Distance (Mpc)} \nonumber \\
D_h(M_c,D_{res}) &:& \textrm{Maximum Resolvable Distance (Mpc)} \nonumber \\
\nonumber
\end{eqnarray}}

Cumulative angle factor refers to the effect source orientation relative to detector angle has on resolvability. Gravitational radiation amplitude has a dependence on angle relative to the polar axis of the binary in question and binaries are oriented randomly relative to our detectors, so not all binaries of the same chirp mass at the same redshift will necessarily have the same resolvability. One binary with chirp mass of 20 $M_{\odot}$ at $z$~=~1 might be resolvable because it is oriented favorably relative to our telescope while another binary with the same parameters might be unresolvable because it is oriented unfavorably. 

$D_L(z)$ and $D_h(M_c,D_{res})$ are further described in \cite{1405.7016}. $D_L(z)$ is, like $P_{ok}$, to complicated to be full described here, but we can describe $D_h(M_c,D_{res})$.

\begin{eqnarray}
D_h(M_c,D_{res}) &=& D_{res} (\frac{M_c}{1.2 M_{\odot}})^{5/6} 
\label{eq:Dh}
\end{eqnarray}

There are two very important things to note about Equation~\ref{eq:Dh}. First is that $D_h$ is dependent on $D_{res}$, which is the maximum distance (in Mpc) at which a theoretical 1 $M_{\odot}$ source would be resolvable. (One can only have a theoretical 1 $M_\odot$ source because the chirp mass for a binary of two minimum mass black holes is roughly 2.5 $M_{\odot}$). Second is that $D_h$ is dependent on chirp mass, which means we need to know the distribution of sources with regards to chirp mass as well as with regards to redshift.

This dependence on $D_{res}$ inherent in $Z_r(z_{max})$ is crucial to our evaluation of $\lambda$, because $D_{res}$ is dependent on the gravitational wave telescope in question. This means we can now consider $\lambda$ to be a function of $D_{res}$, so our tool with which to determine the resolvability of the astrophysical background has a clear telescope dependence (and it's rather important that that be so). For ALIGO $D_{res}$~=~440~Mpc \cite{insert}. In future telescopes we can expect it to be ... !!! waiting on paper from O'Shaughnessy !!!

Before we discuss the mass distribution of sources let us first look at some examples of the distribution of resolvable sources compared to the total distribution of sources for different chirp masses using ALIGO's $D_{res}$, described by Equation~\ref{eq:PDFzrm} and shown in Figures~\ref{fig:PDFzr10} and \ref{fig:PDFzr50}. 

{\small
\begin{eqnarray}
&PDF[Z_r](z,M_c,D_{res}) \nonumber \\&~~~~~= PDF[Z]P_{ok}(z,M_c,D_{res})
\label{eq:PDFzrm}
\end{eqnarray}}

\begin{figure}[!h!t]
	\centering
	\includegraphics[width=3in]{PDFzr10.pdf}
	\caption{$PDF[Z_r](z,10M_{\odot},440Mpc)$ \newline- Equation~\ref{eq:PDFz} and Equation~\ref{eq:PDFzrm}}
	\label{fig:PDFzr10}
\end{figure} 

\begin{figure}[!h!t]
	\centering
	\includegraphics[width=3in]{PDFzr50.pdf}
	\caption{$PDF[Z_r](z,50M_{\odot},440Mpc)$ \newline- Equation~\ref{eq:PDFz} and Equation~\ref{eq:PDFzrm}}
	\label{fig:PDFzr50}
\end{figure} 

Figures~\ref{fig:PDFzr10} and \ref{fig:PDFzr50} represent the extremes of our mass distribution, which is bounded between 10 and 50~$M_{\odot}$. Which brings me, reluctantly, back to what our mass distribution actually is. Frankly, no one knows \cite{insert}. In the future, once much more gravitational wave data has been accrued and when many more black hole binaries have been identified, we may be able to use a well defined function to describe our mass distribution but for now we have nothing better than guesswork. To that end we're going to guess that our mass distribution is of order $\frac{1}{M_c^2}$ which according to \cite{insert} is a reasonable guess. (Note that a proper mass distribution ought to have some $z$ dependence, which our guess fails to account for at all). 

Applying this mass distribution to our PDF of resolvable sources we can derive Equation~\ref{eq:PDFzr} and generate Figure~\ref{fig:PDFzr}. Note however that we do not simply multiply $PDF[Z_r](z,M_c,D_{res}$ by the mass distribution and integrate however, that integral is also divided by integral of the mass distribution. This needs to be done because $PDF[Z_r](z,M_c,D_{res}$ is unitless and integrating over the mass distribution gives units of inverse mass which must be compensated for. 

{\tiny
\beq
PDF[Z_r](z,D_{res}) = \frac{\int_{10M{\odot}}^{50M{\odot}} \frac{PDF_r(Z)(z,M_c,D_{res})}{M_c^2}~dM_c}{\int_{10M{\odot}}^{50M{\odot}} \frac{1}{M_c^2}~dM_c}
\label{eq:PDFzr}
\eeq}

\begin{figure}[!h!t]
	\centering
	\includegraphics[width=3in]{PDFzr.pdf}
	\caption{$PDF[Z_r](z,440Mpc)$ \newline- Equation~\ref{eq:PDFz} and Equation~\ref{eq:PDFzr}}
	\label{fig:PDFzr}
\end{figure} 

From this PDF we can define $CDF[Z_r]$, described in Equation~\ref{eq:CDFzr} and shown in Figure~\ref{fig:CDFzr}. 

{\small
\beq
CDF[Z_r](z,D_{res}) = \int_0^z PDF_r[Z](z,D_{res})~dz
\label{eq:CDFzr}
\eeq}

\begin{figure}[!h!t]
	\centering
	\includegraphics[width=3in]{CDFzr.pdf}
	\caption{$CDF[Z_r](z,440Mpc)$ \newline- Equation~\ref{eq:CDFz} and Equation~\ref{eq:CDFzr}}
	\label{fig:CDFzr}
\end{figure} 



\subsection*{$\bm\lambda_Z$}

We can now use Equation~\ref{eq:CDFz} and Equation~\ref{eq:CDFzr} to find $\lambda_Z$ from Equation~\ref{eq:lambda_parts}, shown in Figure~\ref{fig:lambdaZz}.

\begin{figure}[!h!t]
	\centering
	\includegraphics[width=3in]{resolvable.pdf}
	\caption{$\frac{CDF[Z_r](z,440Mpc)}{CDF[Z](z,440Mpc)}$}
	\label{fig:lambdaZz}
\end{figure}

From here evaluating our CDFs over all relevant parameter space is easy, we simply set $z$ to $z_{max}$~=~6 for Equations~\ref{eq:CDFz} and \ref{eq:CDFzr}. This gives us the $CDF_{tot}$ for both our total and resolvable signal and allows us to calculate $\lambda_Z$ for any $D_{res}$. This relationship between $\lambda_Z$ and $D_{res}$ is shown in Figure~\ref{fig:lambdaZ}, from $D_{res}$ values of 440~Mpc to 4400~Mpc.

\begin{figure}[!h!t]
	\centering
	\includegraphics[width=3in]{lambdaZ.pdf}
	\caption{$\lambda_Z(D_{res}) = \frac{CDF_{tot}[Z_r](6,D_{res})}{CDF_{tot}[Z](6,D_{res})}$}
	\label{fig:lambdaZ}
\end{figure} 

Particular points of note are $\lambda_Z(440Mpc)$ = 0.14, !!! insert points for future detectors here !!!



\subsection*{$\bm{F(f)}$}

Let us consider $F(f)$ (Equation~\ref{eq:frequency}). The first thing we ought to note is a constraint not immediately apparent in the equation itself. Inspiraling black holes have a maximum frequency based on their chirp mass. This maximum frequency is, in geometrized units, the inverse of the chirp mass $\frac{1}{M_c}$~\cite{math.9909058}. If we naively assume that binary black holes simply stop generating a signal at this frequency we get Equation~\ref{eq:frequency_unitstep}, shown for an array of chirp mass values in Figure~\ref{fig:frequency_unitstep}. Note that this is not a probability distribution function yet, as Equation~\ref{eq:frequency_unitstep} still has units of mass.

{\scriptsize
\begin{eqnarray}
F(f,M_c) &=& M_c^{5/3}~(\frac{f}{c})^{2/3}~\mathcal{H}(1-M_c\frac{f}{c})
\label{eq:frequency_unitstep} \\
\mathcal{H} &:& \textrm{Heavyside Function} \nonumber \\
\nonumber
\end{eqnarray}}

(Note that frequency, $f$, is divided by $c$ in Equation~\ref{eq:frequency_unitstep} to convert it into geometrized units so that the equation describing signal energy, $F(f,M_c)$ has the appropriate units of energy, given as chirp mass, $M_c$). 

\begin{figure}[!h!t]
	\centering
	\includegraphics[width=3in]{F_heavyside.pdf}
	\caption{$F(f,M_c)$ - Equation \ref{eq:frequency_unitstep}}
	\label{fig:frequency_unitstep}
\end{figure} 

However using a Heavyside function to describe the frequency limit of inspirals proves to be inaccurate. Much more accurate is the rolloff function described in Equation~\ref{eq:rolloff} and shown in Figure~\ref{fig:frequency_rolloff}, the derivation and calculation of which is described in detail in Appendix \ref{ap:rolloff}. 

{\footnotesize
\begin{eqnarray}
H(f,M_c) &=& (\frac{f}{f_{ref}})^{7/3}~(\frac{|h(f,M_c)|^2}{|h(f_{ref},M_c)|^2})
\label{eq:rolloff} \\
h(f) &:& \textrm{Source Strain} \nonumber
\end{eqnarray}}

\begin{figure}
	\centering
	\includegraphics[width=3in]{rolloff.pdf}
	\caption{$H(f,10,M_c)$ - Equation~\ref{eq:rolloff}}
	\label{fig:frequency_rolloff}
\end{figure}

Applying this rolloff function to Equation~\ref{eq:frequency} instead of a Heavyside function gives us Equation~\ref{eq:frequency_rolloff}, shown in Figure~\ref{fig:frequency_rolloff}. 

{\small
\beq
F(f,M_c) = M_c^{5/3}~(\frac{f}{c})^{2/3}~H(f,M_c)
\label{eq:frequency_rolloff} 
\eeq}

\begin{figure}[!h!t]
	\centering
	\includegraphics[width=3in]{F_rolloff.pdf}
	\caption{$F(f,M_c)$ - Equation \ref{eq:frequency_rolloff}}
	\label{fig:frequency_rolloff}
\end{figure}

Similarly to what we did to transform $PDF[Z_r](z,M_c,D_{res})$ into $PDF[Z_r](z,D_{res})$ we now integrate $F(f,M_c)$ with respect to our approximate mass distribution of binary sources to find $PDF[F](f)$, described in Equation~\ref{eq:PDFf} and shown in Equation~\ref{fig:PDFf}. However unlike with what we did previously, this time we do not need to divide the equation by the integral of the mass distribution to compensate for a units problem. In fact integrating $F(f,M_c)$ over the mass distribution actually solves a units problem, converting $F(f,M_c)$'s units of mass into a unitless quantity, which makes the following equation a true probability distribution function. 

{\scriptsize
\beq
PDF[F](f) = \int_{10M_{\odot}}^{50M_{\odot}} M_c^{5/3}~(\frac{f}{c})^{2/3}~H(f,M_c)~dM_c
\label{eq:PDFf}
\eeq}

\begin{figure}[!h!t]
	\centering
	\includegraphics[width=3in]{PDFf.pdf}
	\caption{$PDF[F](f)$ - Equation \ref{eq:PDFf}}
	\label{fig:PDFf}
\end{figure}

$CDF_{tot}[F]$ can then be found by integrating $PDF[F](f)$ for all frequency values, from 0~Hz to just over 800~Hz, described in Equation~\ref{eq:CDFftot}.

\beq
CDF_{tot}[F] = 50.22
\label{eq:CDFftot}
\eeq



\subsection*{$\bm{F_r(f)}$}

Finding the resolved fraction of sources with respect to frequency is simple. Each gravitational wave telescope has a frequency  dependent resolvability curve, a range of signal amplitudes within which it can resolve sources. We can see these curves compared to the source signal distribution for ALIGO in Figure~\ref{fig:ALIGO}~\cite{1602.03847} and for the upcoming Einstein Telescope in Figure~\ref{fig:Einstein}~\cite{1012.0908}.

\begin{figure}[!h!t]
	\centering
	\includegraphics[width=3in]{ALIGO.pdf}
	\caption{$PDF[F](f)$ - \newline Equation \ref{eq:PDFf} and ALIGO}
	\label{fig:ALIGO}
\end{figure}

\begin{figure}[!h!t]
	\centering
	\includegraphics[width=3in]{Einstein.pdf}
	\caption{$PDF[F](f)$ - \newline Equation \ref{eq:PDFf} and Einstein Telescope}
	\label{fig:Einstein}
\end{figure}

The resolvable portion of signals, with respect to frequency, is simply the area inside both curves. This makes $CDF_{tot}[F_r]$ simple to evaluate, described in Equation~\ref{eq:ALIGO} and Equation~\ref{eq:Einstein}.

{\small
\begin{eqnarray}
\textrm{ALIGO} : CDF_{tot}[F] = 28.89 
\label{eq:ALIGO} \\
\textrm{Einstein Telescope} : CDF_{tot}[F] = 47.68
\label{eq:Einstein}
\end{eqnarray}}



\subsection*{$\bm\lambda_F$}

$\lambda_F$ is easy to evaluate given the results in Equations~\ref{eq:CDFftot}, \ref{eq:ALIGO}, and \ref{eq:Einstein}. Exact values for ALIGO and the Einstein Telescope are shown in Equation~\ref{eq:ALIGO_final} and Equation~\ref{eq:Einstein_final}. Further $\lambda_F$ values for other telescopes should be easy to generate given their resolvability curves.

\begin{eqnarray}
\textrm{ALIGO} : \lambda_F = 0.575
\label{eq:ALIGO_final} \\
\textrm{Einstein Telescope} : \lambda_F = 0.949
\label{eq:Einstein_final}
\end{eqnarray}




\section*{Imperfect Signal Removal}

$\lambda$ only describes the fraction of signal which can be removed in a perfect world where any resolved signal can be made to vanish without leaving behind any trace or affecting the quantity of usable data available. Unfortunately that's not the world we live in. Resolving sources is not the same thing as deleting them. 

In this paper we consider two methods of imperfect signal removal: deletion and subtraction. We modify $\lambda$ to account for the imperfect natures of signal deletion and signal subtraction and then compare them to see which method is more useful for different telescopes. As a metric by which to compare them we analyze how long it would take for a telescope to acquire enough data to reliably confirm the existence of (!!! insert specific model here !!!) cosmic background radiation, using that method. 



\section*{Signal Deletion}

Signal deletion is a very simple method for removing resolved signals. Any data which contains the signal of a resolvable source is deleted. This perfectly removes resolved signals but it comes at a price, after deletion one has less data to work with. 

There is a complexity in determining the efficacy of this method though. The length of time in which a source's signal is resolvable depends on the frequency of the signal. This makes determining the quantity of data that needs to be deleted to remove all resolved sources a non-trivial task. Solving this problem will be done as part of Capstone II. 



\section*{Signal Subtraction}

Signal subtraction is a more complicated method of removing resolved signals. Resolved signals are analyzed to determine, within some margin of error, the parameters of the source which created the signal. These are then used to simulate a close facsimile of the signal which is subtracted from the data. The problem with this method is the residual difference between the simulated and actual signal which is left behind. This residual acts as an unresolved signal in its own right and contributes to the astrophysical background. 

Determining the efficacy of this method for an arbitrary detector is very complicated and doing so will be the main thrust of Capstone II. 



\clearpage

\section*{Appendices}







\subsection{Rolloff Function}
\label{ap:rolloff}

Using a Heavyside function to describe the frequency range of a black hole binary is equivalent to saying that a binary simply ceases to emit a gravitational wave signal at a certain frequency because it has reached the minimum radius of its inspiral and that nothing else of note affects its radiated energy as it inspirals. This of course is a very unphysical way of describing gravitational wave energy. 

A better way to describe how the signal of a gravitational wave source, in this case an inspiraling black hole binary, changes with frequency is with its flux, $\Phi(f)$. Flux has units of inverse area which, in geometrized units, is equivalent to mass per volume. Therefore a source's flux can be thought of as the energy density, or signal density, of that source. 

This means that the ratio of the volumetric integral of a source's flux at a given frequency, $\Phi(f)$, compared to the volumetric integral of that source's flux at a reference frequency, $\Phi(f_{ref})$, can be thought of as the ratio of energy, or signal amplitude, that a source emits at a given frequency compared to the reference frequency. By taking our reference frequency to be the lowest frequency detectable by our gravitational wave telescope (roughly 10~Hz for ALIGO \cite{1602.03847} and roughly 2~Hz for the upcoming Einstein Telescope~\cite{1607.08697}) we can naively find a ratio which describes how a source's radiated energy varies with frequency, described in Equation~\ref{eq:flux_ratio}.

{\footnotesize
\beq
\textrm{Naive Signal Amplitude Ratio} = \frac{\int\Phi(f)~dr^3}{\int\Phi(f_{ref})~dr^3}
\label{eq:flux_ratio}
\eeq}

This has some problems though. The volumetric integrals in that ratio severely complicate any equations involving this ratio. So to simplify such equations we would like to find a unitless component of flux which can be extracted from each integral and compare the ratios of those components instead of the flux integrals as a whole. 

From~\cite{gravitation} we get Equation~\ref{eq:proportions}, showing the proportionality of flux to strain, $h(f)$.

\beq
\Phi(f) \propto \frac{\partial E}{\partial f} \propto f^2~|h(f)|^2
\label{eq:proportions}
\eeq 

Strain squared, $|h(f)|^2$, is a suitable component to pull out of flux but it is not unitless. So to make it unitless we must find what the strain squared is proportional to (in terms of frequency, $f$) and multiply $|h(f)|^2$ by the inverse of that to make it unitless. To find this proportionality we use Kepler's laws of orbital mechanics for circular motion~\ref{eq:kepler}, which are a reasonable approximation to make for the majority of a binary's inspiral, assuming that the black holes which comprise the binary have equal mass. Which is an assumption we will make. (!!! Future research may revolve around removing this assumption !!!) 

\begin{eqnarray}
E &=& -\frac{M^2}{2a} \label{eq:kepler} \\
2 \pi f_{orb} &=& (\frac{M}{a^3})^{1/2} \nonumber \\
E &:& \textrm{Gravitational Potential Energy} \nonumber \\
M &:& \textrm{Orbiting Bodies' Masses} \nonumber \\
a &:& \textrm{Orbital Acceleration} \nonumber \\
f_{orb} &:& \textrm{Orbital Frequency} \nonumber \\
\end{eqnarray}

(Note that Kepler's orbital frequency is the same thing as signal frequency for gravitational waves). 

From Equation~\ref{eq:kepler} we can derive Equation~\ref{eq:energy} and take its derivative with respect to frequency, $f$, to find Equation~\ref{eq:flux}. 

\begin{eqnarray}
E &=& -\frac{M}{2}(M \pi f)^{2/3}
\label{eq:energy} \\ 
\frac{\partial E}{\partial f} &=& \frac{M^2 \pi}{3}(M \pi f)^{-1/3}
\label{eq:flux}
\end{eqnarray}

This allows us to say to $\frac{\partial E}{\partial f} \propto f^{1/3}$ which when combined with Equation~\ref{eq:proportions} allows us to generate Equation~\ref{eq:fhf} and from that find Equation~\ref{eq:hf}.

\begin{eqnarray}
f^{-1/3} \propto f^2~|h(f)|^2
\label{eq:fhf} \\
|h(f)|^2 \propto f^{-7/3}
\label{eq:hf} 
\end{eqnarray}

Therefore to make strain squared, $|h(f)|^2$, unitless we multiply it by $f^{7/3}$. Pulling this unitless component of flux out of the numerator and denominator of Equation~\ref{eq:flux_ratio} and taking that ratio as our signal amplitude ratio we finally get Equation~\ref{eq:ratio}, which is what we want to describe our rolloff function.

{\small
\beq
\textrm{Signal Amplitude Ratio} = \frac{f^{7/3}~|h(f)|^2}{f_{ref}^{7/6}~|h(f_{ref})|^2}
\label{eq:ratio}
\eeq}

However strain is not just a factor of frequency, $f$, it is also a factor of chirp mass, $M_c$. So for our calculations we will be using the equations for binary strain, $h(f)$, with respect to chirp mass, $M_c$, developed in~\cite{0909.2867}. This allows us to calculate our rolloff equation shown earlier in Equation~\ref{eq:rolloff}.




\clearpage

\begin{thebibliography}{100}

	\bibitem{1602.03837} The LIGO Scientific Collaboration and  the Virgo Collaboration.
\\ Observation of Gravitational Waves from a Binary Black Hole Merger, 2016,
\\ Phys. Rev. Lett. 116, 061102 (2016);
\\ arXiv:1602.03837.
\\ DOI: 10.1103/PhysRevLett.116.061102.

	\bibitem{0912.1074} Ilya Mandel and Richard O'Shaughnessy.
\\ Compact Binary Coalescences in the Band of Ground-based Gravitational-Wave Detectors, 2009,
\\ Class.Quant.Grav.27:114007,2010;
\\ arXiv:0912.1074.
\\ DOI: 10.1088/0264-9381/27/11/114007.

	\bibitem{1604.02513} Thomas Callister, Letizia Sammut, Eric Thrane, Shi Qiu and Ilya Mandel.
\\ The limits of astrophysics with gravitational wave backgrounds, 2016;
\\ arXiv:1604.02513.

	\bibitem{1206.1330} Pablo A. Rosado.
\\ Gravitational wave background from rotating neutron stars, 2012,
\\ Physical Review D 86, 104007 (2012);
\\ arXiv:1206.1330.
\\ DOI: 10.1103/PhysRevD.86.104007.

	\bibitem{Marranghello} G. F. Marranghello.
\\ Phase Transition In Rotating Neutron Stars
\\ International Journal of Modern Physics D 2007 16:02n03, 333-339

	\bibitem{1106.2736} G. F. Burgio, V. Ferrari, L. Gualtieri and H. J. Schulze.
\\ Oscillations of hot, young neutron stars: Gravitational wave frequencies and damping times, 2011,
\\ Phys.Rev.D84:044017,2011;
\\ arXiv:1106.2736.
\\ DOI: 10.1103/PhysRevD.84.044017.

	\bibitem{astro-ph/0412277} Alessandra Buonanno, Guenter Sigl, Georg G. Raffelt, Hans-Thomas Janka and Ewald Mueller.
\\ Stochastic Gravitational Wave Background from Cosmological Supernovae, 2004,
\\ Phys.Rev. D72 (2005) 084001;
\\ arXiv:astro-ph/0412277.
\\ DOI: 10.1103/PhysRevD.72.084001.

	\bibitem{hep-th/9211021} M. Gasperini and G. Veneziano.
\\ Pre-Big-Bang in String Cosmology, 1992,
\\ Astropart.Phys.1:317-339,1993;
\\ arXiv:hep-th/9211021.
\\ DOI: 10.1016/0927-6505(93)90017-8.

	\bibitem{hep-th/9701146} A. Buonanno, M. Gasperini, M. Maggiore and C. Ungarelli.
\\ Expanding and contracting universes in third quantized string cosmology, 1997,
\\ Class.Quant.Grav.14:L97-L103,1997;
\\ arXiv:hep-th/9701146.
\\ DOI: 10.1088/0264-9381/14/5/005.

	\bibitem{1006.0217} Jean-Francois Dufaux, Daniel G. Figueroa and Juan Garcia-Bellido.
\\ Gravitational Waves from Abelian Gauge Fields and Cosmic Strings at Preheating, 2010,
\\ Phys.Rev.D82:083518,2010;
\\ arXiv:1006.0217.
\\ DOI: 10.1103/PhysRevD.82.083518.

	\bibitem{Grishchuk1} L.P. Grishchuk.
\\ Amplification of gravitational waves in an isotropic universe
\\ JETP, 1975, Vol. 40, No. 3, p. 409
\\ \url{http://www.jetp.ac.ru/cgi-bin/dn/e_040_03_0409.pdf}

	\bibitem{Grishchuk2} L. P. Grishchuk.
\\ Relic gravitational waves and limits on inflation
\\ Phys. Rev. D 48, 3513 – Published 15 October 1993
\\ \url{http://journals.aps.org/prd/pdf/10.1103/PhysRevD.48.3513}

	\bibitem{0711.2593} Chiara Caprini, Ruth Durrer and Geraldine Servant.
\\ Gravitational wave generation from bubble collisions in first-order phase transitions: an analytic approach, 2007,
\\ Phys.Rev.D77:124015,2008;
\\ arXiv:0711.2593.
\\ DOI: 10.1103/PhysRevD.77.124015.

	\bibitem{0909.0622} Chiara Caprini, Ruth Durrer and Geraldine Servant.
\\ The stochastic gravitational wave background from turbulence and magnetic fields generated by a first-order phase transition, 2009,
\\ JCAP 0912:024,2009;
\\ arXiv:0909.0622.
\\ DOI: 10.1088/1475-7516/2009/12/024.

	\bibitem{0901.1661} Chiara Caprini, Ruth Durrer, Thomas Konstandin and Geraldine Servant.
\\ General Properties of the Gravitational Wave Spectrum from Phase Transitions, 2009,
\\ Phys.Rev.D79:083519,2009;
\\ arXiv:0901.1661.
\\ DOI: 10.1103/PhysRevD.79.083519.

	\bibitem{hep-th/0410222} Thibault Damour and Alexander Vilenkin.
\\ Gravitational radiation from cosmic (super)strings: bursts, stochastic background, and observational windows, 2004,
\\ Phys.Rev. D71 (2005) 063510;
\\ arXiv:hep-th/0410222.
\\ DOI: 10.1103/PhysRevD.71.063510.

	\bibitem{1106.5795} Pablo A. Rosado.
\\ Gravitational wave background from binary systems, 2011,
\\ Physical Review D 84, 084004 (2011);
\\ arXiv:1106.5795.
\\ DOI: 10.1103/PhysRevD.84.084004.

	\bibitem{1602.03839} The LIGO Scientific Collaboration,  the Virgo Collaboration
\\ GW150914: First results from the search for binary black hole coalescence with Advanced LIGO, 2016,
\\ Phys. Rev. D 93, 122003 (2016);
\\ arXiv:1602.03839.
\\ DOI: 10.1103/PhysRevD.93.122003.

	\bibitem{1112.1898} Chengjian Wu, Vuk Mandic and Tania Regimbau.
\\ Accessibility of the Gravitational-Wave Background due to Binary Coalescences to Second and Third Generation Gravitational-Wave Detectors, 2011;
\\ arXiv:1112.1898.
\\ DOI: 10.1103/PhysRevD.85.104024.

	\bibitem{1602.03847} The LIGO Scientific Collaboration and  the Virgo Collaboration.
\\ GW150914: Implications for the stochastic gravitational wave background from binary black holes, 2016,
\\ Phys. Rev. Lett. 116, 131102 (2016);
\\ arXiv:1602.03847.
\\ DOI: 10.1103/PhysRevLett.116.131102.

	\bibitem{1602.04531} Krzysztof Belczynski, Daniel E. Holz, Tomasz Bulik and Richard O'Shaughnessy.
\\ The first gravitational-wave source from the isolated evolution of two 40-100 Msun stars, 2016;
\\ arXiv:1602.04531.
\\ DOI: 10.1038/nature18322.

	\bibitem{1405.7016} M. Dominik, E. Berti, R. O'Shaughnessy, I. Mandel, K. Belczynski, C. Fryer, D. Holz, T. Bulik and F. Pannarale.
\\ Double Compact Objects III: Gravitational Wave Detection Rates, 2014;
\\ arXiv:1405.7016.
\\ DOI: 10.1088/0004-637X/806/2/263.

	\bibitem{math.9909058} Ivan Mirkovic and Dmitriy Rumynin.
\\ Geometric Representation Theory of Restricted Lie Algebras of Classical Type, 1999;
\\ arXiv:math/9909058.

	\bibitem{1012.0908} S Hild, et al. 
\\ Sensitivity Studies for Third-Generation Gravitational Wave Observatories, 2010;
\\ arXiv:1012.0908.
\\ DOI: 10.1088/0264-9381/28/9/094013.

	\bibitem{1607.08697} LIGO Scientific Collaboration
\\ Exploring the Sensitivity of Next Generation Gravitational Wave Detectors, 2016;
\\ arXiv:1607.08697.

	\bibitem{gravitation} Misner, C.W. and Thorne, K.S. and Wheeler, J.A.
\\ Gravitation, 1973
\\ ISBN:9780716703440.
\\ lCCN:78156043.

	\bibitem{0909.2867} P. Ajith, M. Hannam, S. Husa, Y. Chen, B. Bruegmann, N. Dorband, D. Mueller, F. Ohme, D. Pollney, C. Reisswig, L. Santamaria and J. Seiler.
\\ Inspiral-merger-ringdown waveforms for black-hole binaries with non-precessing spins, 2009,
\\ Phys.Rev.Lett.106:241101,2011;
\\ arXiv:0909.2867.
\\ DOI: 10.1103/PhysRevLett.106.241101.

	\bibitem{astro-ph/0507028} J. C. N. de Araujo and O. D. Miranda.
\\ Star formation rate density and the stochastic background of gravitational waves, 2005,
\\ Phys.Rev.D71:127503,2005;
\\ arXiv:astro-ph/0507028.
\\ DOI: 10.1103/PhysRevD.71.127503.


\end{thebibliography}
\end{document}































