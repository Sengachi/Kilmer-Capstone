\documentclass[twocolumn,11pt]{article}
\setlength{\textheight}{9truein}
\setlength{\topmargin}{-0.9truein}
\setlength{\parindent}{0pt}
\setlength{\parskip}{10pt}
\setlength{\columnsep}{.4in}
%
\newcommand{\beq}{\begin{equation}}
\newcommand{\eeq}{\end{equation}}
%
\usepackage{graphicx}
\usepackage{epsfig}
\usepackage{url}
\usepackage{cuted}
\usepackage{appendix}
\usepackage{bm}
%
\title{Assessing the Astrophysical Gravitational Wave Foreground \\ \vspace{.1in} \large Comparing the Efficacies of Signal Deletion and Signal Subtraction with Respect to Gravitational Wave Telescope Parameters}
\author{Thomas Kilmer}
\date{18 Nov 2016}
%
%
\begin{document}
 \twocolumn[
   \begin{@twocolumnfalse}
   \maketitle
   \setlength{\parindent}{0pt}
   \begin{abstract} 

Gravitational waves from astrophysical sources have been detected and the next generation of gravitational wave telescopes might be able to detect gravitational waves from cosmic background sources. However we expect that these cosmic background signals will be obscured by a substantial astrophysical background which will make analyzing the cosmic background prohibitively difficult. In this paper we assess techniques for removing resolved astrophysical sources from gravitational wave data for the purpose of reducing the astrophysical background and easing analysis of the cosmic gravitational wave background. 
   
  \vspace{.3in} 
     \end{abstract}
    \end{@twocolumnfalse}]

\section*{Introduction}

Advanced LIGO recently detected gravitational waves from a binary black hole merger~\cite{1602.03837}. While estimates for the number of coalescing binary black holes in the universe are imprecise, there are many such binaries~\cite{0912.1074}. This large number of binary black holes produces overlapping, incoherent signals which form a ``background'' of gravitational waves~\cite{1604.02513}. This background is statistically homogeneous and isotropic, despite the discrete nature of the sources which make it up~\cite{1604.02513}. With current advances in gravitational wave detection this background will be measured within the next few years~\cite{1604.02513}.

!!! Insert description of background here. !!!

Binary black holes are not the only thing which contribute to the gravitational wave background. A number of other astrophysical sources contribute to the background including rotating neutron stars~\cite{1206.1330}, neutron star phase transitions~\cite{Marranghello}, instabilities in young neutron stars~\cite{1106.2736}, and core collapse supernovae~\cite{astro-ph/0412277}. There are also cosmological sources which contribute to the gravitational wave background including pre-Big Bang cosmological factors~\cite{hep-th/9211021}~\cite{hep-th/9701146}~\cite{1006.0217}, the amplification of weak gravitational waves (including vacuum fluctuations) during inflation~\cite{Grishchuk1}~\cite{Grishchuk2}, phase transitions in the early universe~\cite{0711.2593}~\cite{0909.0622}~\cite{0901.1661}, and cosmic strings~\cite{hep-th/0410222}. The analysis of gravitational waves produced by the aforementioned cosmological sources could provide insight into the early processes of the universe.

However the analysis of the gravitational wave background produced by cosmological sources is complicated by the inability to separate it from the gravitational wave background produced by astrophysical sources~\cite{1604.02513}. However the astrophysical background can be separated into two parts, an unresolvable astrophysical background in which every source generates signals with signal to noise ratios less than 1 and a resolvable astrophysical background in which every source generates signals with SNRs greater than or equal to 1 called the astrophysical foreground~\cite{1106.5795}. The signals generated by the astrophysical foreground can then be identified and removed from the data, reducing the total astrophysical background and making it easier to identify any cosmic background which may exist. 

This project is not concerned with how the astrophysical foreground signals are identified, which has been discussed in great detail elsewhere~\cite{insert}. The goal of this project is instead to develop tools for removing the signals generated by the astrophysical foreground. Before doing so however we need to develop a metric for determining the efficacy of such tools, which is what we have been doing during Capstone I. 

\section*{Astrophysical Background Signal Density}

Let us begin with the signal density for astrophysical gravitational waves described in~\cite{1112.1898} (Equation~\ref{eq:signal_density}). Note that this formula only accounts for the gravitational waves produced by binary black hole mergers, though it is still a reasonable model for the total astrophysical background as black hole binary mergers are expected to constitute the vast majority of all astrophysical gravitational wave sources~\cite{insert}. 

{\tiny
\beq
\Omega_{g \omega} = \frac{8 \lambda (\pi G M_c)^{5/3}}{9 H_0^3 c^2} f^{2/3} \int_{0}^{z_{max}} \frac{R_V(z)~dz}{(1+z)^{1/3} E[\Omega_M, \Omega_{\Lambda}, z]}
\label{eq:signal_density} 
\eeq
}{\footnotesize
\begin{eqnarray}
R_v(z) &:& \textrm{Observed~Rate~of~Binary~Coalescence} \nonumber \\
E[\Omega_M, \Omega_{\Lambda}, z] &:& \textrm{z-Dependence~of~Comoving~Volume}  \nonumber  \\
(1+z)^{1/3} &:& \textrm{Comoving~Volume~Term} \nonumber \\
z &:& \textrm{Source Redshift} \nonumber \\
z_{max} &:& \textrm{Most Distant Coalescing Binaries} \nonumber \\
\frac{8 \lambda (\pi G)^{5/3}}{9 H_0^3 c^2} &:& \textrm{Constant~Factors} \nonumber \\
M_c &:& \textrm{Binary Chirp Mass} \nonumber \\
f &:& \textrm{Signal Frequency} \nonumber \\
\nonumber
\end{eqnarray}
}

To simplify how we model $\Omega_{gw}$ we can split it into three parts, its constant terms, frequency related terms, and redshift related terms. Keep in mind that the frequency of an orbit, an thus the frequency of a gravitational wave signal, is dependent on the masses of the orbiting objects and so chirp mass is a frequency related term. 

{\footnotesize
\begin{eqnarray}
\Omega_{gw} &=& C~F(f)~Z(z_{max}) 
\label{eq:parts}\\
C &=&  \frac{8 \lambda (\pi G)^{5/3}}{9 H_0^3 c^2} 
\label{eq:constants}\\
F(f) &=& M_c^{5/3} f^{2/3} 
\label{eq:frequency}\\
Z(z_{max}) &=& \int_{0}^{z_{max}}\frac{R_V(z)~dz}{(1+z)^{1/3} E[\Omega_M, \Omega_{\Lambda}, z]} 
\label{eq:redshift}
\end{eqnarray}}

\section*{Resolving the Astrophysical Foreground}

The success of a technique used to resolve sources can be quantified by what fraction of the background it can resolve, how much of the background it can convert into foreground. That is, a good technique has a $\lambda$ close to 1 (Equation~\ref{eq:lambda}). 

\begin{eqnarray}
\lambda &=& \frac{CDF(\Omega_{gw,r})}{CDF(\Omega_{gw})} 
\label{eq:lambda} \\
\Omega_{gw,r} &:& \textrm{Resolvable Signal} \nonumber \\
\Omega_{gw} &:& \textrm{Total Signal} \nonumber \\
CDF &:& \textrm{Cumulative Distribution Function} \nonumber \\
\nonumber
\end{eqnarray}

This can then be broken down into component parts, exactly as in Equation~\ref{eq:parts}, which we show in Equation~\ref{eq:lambda_parts}. (Note that C, Equation~\ref{eq:constants}, cancels out in the numerator and the denominator).

{\footnotesize
\begin{eqnarray}
\lambda &=& \lambda_F~\lambda_Z 
\label{eq:lambda_parts} \\
\lambda_F &=& \frac{CDF[F_r]}{CDF[F]} \nonumber \\
\lambda_Z &=& \frac{CDF[Z_r]}{CDF[Z]} \nonumber \\
F_r &:& \textrm{Resolvable Portion of $F$} \nonumber \\
Z_r &:& \textrm{Resolvable Portion of $Z$} \nonumber \\
\nonumber
\end{eqnarray}}



\subsection*{$\bm{F(f)}$}

\subsection*{$\bm{F_r(f)}$}

\subsection*{$\bm{Z(z_{max})}$}

Let us consider Equation~\ref{eq:redshift} and its component parts. 

{\footnotesize
\begin{eqnarray}
Z(z_{max}) &=& \int_{0}^{z_{max}}\frac{R_V(z)~dz}{(1+z)^{1/3} E[\Omega_M, \Omega_{\Lambda}, z]} \nonumber \\
R_V(z) &:& \textrm{Observed Rate of Binary Coalescence} \nonumber \\
E[\Omega_M, \Omega_{\Lambda}, z] &:& \textrm{z-Dependence of Comoving Volume} \nonumber \\
(1+z)^{1/3} &:& \textrm{Comoving Volume Factor} \nonumber \\
z_{max} &:& \textrm{Most Distant Coalescing Binaries} \nonumber
\nonumber
\end{eqnarray}}

$R_V(z)$ is determined by observations of gravitational wave sources and is thus not well defined, given the small number of sources so far observed. For the sake of this paper we will be using the theoretical formula for $R_V(z)$ described in~\cite{1602.04531}. $E[\Omega_M, \Omega_{\Lambda}, z]$ and $z_{max}$ are defined in~\cite{1112.1898}. 

\begin{eqnarray}
R_V(z) &=&  0.015~\frac{(1+z)^{2.7}}{1 + [\frac{1+z}{2.9}]^{5.6}} \frac{M_{\odot}}{Mpc^3yr} \nonumber \\
E[\Omega_M, \Omega_{\Lambda}, z] &=& \sqrt{\Omega_M (1+z^3) + \Omega_{\Lambda}} \nonumber \\
z_{max} &=& 6 \nonumber 
\end{eqnarray}

Using standard $\Lambda$CDM cosmology's values of $\Omega_M$~=~0.3 and $\Omega_{\Lambda}$~=~0.7 this allows us to describe Equation~\ref{eq:redshift} in detail, shown and evaluated in Equation~\ref{eq:redshift_numbers}. 

{\footnotesize
\begin{eqnarray}
Z(z_{max}) &=& \int_{0}^{6}\frac{0.015~(1+z)^{2.37}~dz}{\sqrt{0.7+0.3(1+z)^3}~(1+\frac{(1+z)^{5.6}}{2.9})}
\nonumber \\
Z(z_{max}) &=& 0.100218 \nonumber \\
\label{eq:redshift_numbers}
\end{eqnarray}}

This in turn allows us to determine the probability distribution function of $Z(z_{max})$. The PDF of a definite integral is described as the integrand divided by the integral and we now know the value of that integral. $PDF(Z(z_{max})$ is evaluated in Equation~\ref{eq:PDFz} and depicted in Figure~\ref{fig:PDFz}. This describes the distribution of black hole binaries by their redshifts. 

{\tiny
\beq
PDF[Z](z) = \frac{0.1497~(1+z)^{2.37}}{\sqrt{0.7+0.3(1+z)^3}~(1+\frac{(1+z)^{5.6}}{2.9})}
\label{eq:PDFz}
\eeq}

\begin{figure}[!h!t]
	\centering
	\includegraphics[width=3in]{PDFz.pdf}
	\caption{$PDF[Z](z)$ - Equation~\ref{eq:PDFz}}
	\label{fig:PDFz}
\end{figure} 

From here it is simple to find the CDF of $Z(z_{max})$. The CDF of a PDF is simply the PDF integrated from the PDF's minimum value to an arbitrary value, in this case integrating from 0 to $z$. $CDF[Z]$ is evaluated in Equation~\ref{eq:CDFz} and depicted in Figure~\ref{fig:CDFz}. This describes the fraction of black hole binaries one would observe within a given redshift. 

{\tiny
\beq
CDF[Z](z) = \int_{0}^{z}\frac{0.1497~(1+z)^{2.37}~dz}{\sqrt{0.7+0.3(1+z)^3}~(1+\frac{(1+z)^{5.6}}{2.9})}
\label{eq:CDFz}
\eeq}

\begin{figure}[!h!t]
	\centering
	\includegraphics[width=3in]{CDFz.pdf}
	\caption{$CDF[Z](z)$ - Equation~\ref{eq:CDFz}}
	\label{fig:CDFz}
\end{figure} 

To make sure we haven't done something horribly wrong we can then check that $CDF[Z](z)$ goes to 1 at $z_{max}$, which it does. 

\subsection*{$\bm{Z_r(z_{max})}$}

Now that we have a numerical description of $CDF[Z]$ we can work on finding a numerical description of $CDF[Z_r]$. The first thing we need to do to make that happen is find an equation that describes the resolvability of sources with respect to redshift. For that we turn to \cite{1405.7016}, which describes in detail Equation~\ref{eq:Pok}. (Unfortunately $P_{ok}$ is too complicated to be fully described here). 

{\footnotesize
\begin{eqnarray}
P_{ok}(\frac{D_L(z)}{D_h(M_c)}) &:& \textrm{Cumulative Angle Factor}
\label{eq:Pok} \\
D_L(z) &:& \textrm{Luminosity Distance (Mpc)} \nonumber \\
D_h(M_c,D_{res}) &:& \textrm{Maximum Resolvable Distance (Mpc)} \nonumber \\
\nonumber
\end{eqnarray}}

Cumulative angle factor refers to the effect source orientation relative to detector angle has on resolvability. Gravitational radiation amplitude has a dependence on angle relative to the polar axis of the binary in question and binaries are oriented randomly relative to our detectors, so not all binaries of the same chirp mass at the same redshift will necessarily have the same resolvability. One binary with chirp mass of 20 $M_{\odot}$ at $z$~=~1 might be resolvable because it is oriented favorably relative to our telescope while another binary with the same parameters might be unresolvable because it is oriented unfavorably. 

$D_L(z)$ and $D_h(M_c,D_{res})$ are further described in \cite{1405.7016}. $D_L(z)$ is, like $P_{ok}$, to complicated to be full described here, but we can describe $D_h(M_c,D_{res})$.

\begin{eqnarray}
D_h(M_c,D_{res}) &=& D_{res} (\frac{M_c}{1.2 M_{\odot}})^{5/6} 
\label{eq:Dh}
\end{eqnarray}

There are two very important things to note about Equation~\ref{eq:Dh}. First is that $D_h$ is dependent on $D_{res}$, which is the maximum distance (in Mpc) at which a theoretical 1 $M_{\odot}$ source would be resolvable. (One can only have a theoretical 1 $M_\odot$ source because the chirp mass for a binary of two minimum mass black holes is roughly 2.5 $M_{\odot}$). Second is that $D_h$ is dependent on chirp mass, which means we need to know the distribution of sources with regards to chirp mass as well as with regards to redshift.

This dependence on $D_{res}$ inherent in $Z_r(z_{max})$ is crucial to our evaluation of $\lambda$, because $D_{res}$ is dependent on the gravitational wave telescope in question. This means we can now consider $\lambda$ to be a function of $D_{res}$, so our tool with which to determine the resolvability of the astrophysical background has a clear telescope dependence (and it's rather important that that be so). For ALIGO $D_{res}$~=~440~Mpc \cite{insert}. In future telescopes we can expect it to be ... !!! waiting on paper from O'Shaughnessy !!!

Before we discuss the mass distribution of sources let us first look at some examples of the distribution of resolvable sources compared to the total distribution of sources for different chirp masses using ALIGO's $D_{res}$, described by Equation~\ref{eq:PDFzrm} and shown in Figures~\ref{fig:PDFzr10} and \ref{fig:PDFzr50}. 

{\small
\begin{eqnarray}
&PDF[Z_r](z,M_c,D_{res}) \nonumber \\&~~~~~= PDF[Z]P_{ok}(z,M_c,D_{res})
\label{eq:PDFzrm}
\end{eqnarray}}

\begin{figure}[!h!t]
	\centering
	\includegraphics[width=3in]{PDFzr10.pdf}
	\caption{$PDF[Z_r](z,10M_{\odot},440Mpc)$ \newline- Equation~\ref{eq:PDFz} and Equation~\ref{eq:PDFzrm}}
	\label{fig:PDFzr10}
\end{figure} 

\begin{figure}[!h!t]
	\centering
	\includegraphics[width=3in]{PDFzr50.pdf}
	\caption{$PDF[Z_r](z,50M_{\odot},440Mpc)$ \newline- Equation~\ref{eq:PDFz} and Equation~\ref{eq:PDFzrm}}
	\label{fig:PDFzr50}
\end{figure} 

Figures~\ref{fig:PDFzr10} and \ref{fig:PDFzr50} represent the extremes of our mass distribution, which is bounded between 10 and 50~$M_{\odot}$. Which brings me, reluctantly, back to what our mass distribution actually is. Frankly, no one knows \cite{insert}. In the future, once much more gravitational wave data has been accrued and when many more black hole binaries have been identified, we may be able to use a well defined function to describe our mass distribution but for now we have nothing better than guesswork. To that end we're going to guess that our mass distribution is of order $\frac{1}{M_c^2}$ which according to \cite{insert} is a reasonable guess. (Note that a proper mass distribution ought to have some $z$ dependence, which our guess fails to account for at all). 

Applying this mass distribution to our PDF of resolvable sources we can derive Equation~\ref{eq:PDFzr} and generate Figure~\ref{fig:PDFzr}.

{\tiny
\beq
PDF[Z_r](z,D_{res}) = \frac{\int_{10M{\odot}}^{50M{\odot}} \frac{PDF_r(Z)(z,M_c,D_{res})}{M_c^2}~dM_c}{\int_{10M{\odot}}^{50M{\odot}} \frac{1}{M_c^2}~dM_c}
\label{eq:PDFzr}
\eeq}

\begin{figure}[!h!t]
	\centering
	\includegraphics[width=3in]{PDFzr.pdf}
	\caption{$PDF[Z_r](z,440Mpc)$ \newline- Equation~\ref{eq:PDFz} and Equation~\ref{eq:PDFzr}}
	\label{fig:PDFzr}
\end{figure} 

From this PDF we can define $CDF[Z_r]$, described in Equation~\ref{eq:CDFzr} and shown in Figure~\ref{fig:CDFzr}. 

{\small
\beq
CDF[Z_r](z,D_{res}) = \int_0^z PDF_r[Z](z,D_{res})~dz
\label{eq:CDFzr}
\eeq}

\begin{figure}[!h!t]
	\centering
	\includegraphics[width=3in]{CDFzr.pdf}
	\caption{$CDF[Z_r](z,440Mpc)$ \newline- Equation~\ref{eq:CDFz} and Equation~\ref{eq:CDFzr}}
	\label{fig:CDFzr}
\end{figure} 

This can in turn be used to find $\lambda_Z$ from Equation~\ref{eq:lambda_parts}, shown in Figure~\ref{fig:lambdaZz}.

\begin{figure}[!h!t]
	\centering
	\includegraphics[width=3in]{resolvable.pdf}
	\caption{$\lambda_Z(z,440Mpc) = \frac{CDF[Z_r](z,440Mpc)}{CDF[Z](z,440Mpc)}$}
	\label{fig:lambdaZz}
\end{figure} 

Of course Figure~\ref{fig:lambdaZz} is somewhat misleading. We can't actually select what signals we want to accept into our data based on whether or not their sources are within a given redshift. We must accept all signals regardless of redshift. Fortunately this actually makes our calculations a good deal easier. We can simply set $z$ for our CDFs to $z_{max}$ and in doing so take all signals into account regardless of redshift and eliminate an independent variable that $\lambda_Z$ is dependent on so that $\lambda_Z$ is only dependent on $D_{res}$. This relationship between $\lambda_Z$ and $D_{res}$ is shown in Figure~\ref{fig:lambdaZ}, from $D_{res}$ values of 440~Mpc to 4400~Mpc.

\begin{figure}[!h!t]
	\centering
	\includegraphics[width=3in]{lambdaZ.pdf}
	\caption{$\lambda_Z(D_{res}) = \frac{CDF[Z_r](6,D_{res})}{CDF[Z](6,D_{res})}$}
	\label{fig:lambdaZ}
\end{figure} 

Particular points of note are $\lambda_Z(440Mpc)$ = 0.14, !!! insert points for future detectors here !!!








\pagebreak

\begin{thebibliography}{1}

\bibitem{1112.1898}
Chengjian Wu, Vuk Mandic and Tania Regimbau.
\newblock Accessibility of the Gravitational-Wave Background due to Binary Coalescences to Second and Third Generation Gravitational-Wave Detectors, 2011;
\newblock arXiv:1112.1898.
\newblock DOI: 10.1103/PhysRevD.85.104024.

\bibitem{1602.04531}
Krzysztof Belczynski, Daniel E. Holz, Tomasz Bulik and Richard O'Shaughnessy.
\newblock The first gravitational-wave source from the isolated evolution of two 40-100 Msun stars, 2016;
\newblock arXiv:1602.04531.
\newblock DOI: 10.1038/nature18322.

\bibitem{1405.7016}
M. Dominik, E. Berti, R. O'Shaughnessy, I. Mandel, K. Belczynski, C. Fryer, D. Holz, T. Bulik and F. Pannarale.
\newblock Double Compact Objects III: Gravitational Wave Detection Rates, 2014;
\newblock arXiv:1405.7016.
\newblock DOI: 10.1088/0004-637X/806/2/263.

\bibitem{1602.03837}
 The LIGO Scientific Collaboration and  the Virgo Collaboration.
\\ Observation of Gravitational Waves from a Binary Black Hole Merger, 2016,
\\ Phys. Rev. Lett. 116, 061102 (2016);
\\ arXiv:1602.03837.
\\ DOI: 10.1103/PhysRevLett.116.061102.

\bibitem{0912.1074}
Ilya Mandel and Richard O'Shaughnessy.
\\ Compact Binary Coalescences in the Band of Ground-based Gravitational-Wave Detectors, 2009,
\\ Class.Quant.Grav.27:114007,2010;
\\ arXiv:0912.1074.
\\ DOI: 10.1088/0264-9381/27/11/114007.

\bibitem{1604.02513}
Thomas Callister, Letizia Sammut, Eric Thrane, Shi Qiu and Ilya Mandel.
\\ The limits of astrophysics with gravitational wave backgrounds, 2016;
\\ arXiv:1604.02513.

\bibitem{1206.1330}
Pablo A. Rosado.
\\ Gravitational wave background from rotating neutron stars, 2012,
\\ Physical Review D 86, 104007 (2012);
\\ arXiv:1206.1330.
\\ DOI: 10.1103/PhysRevD.86.104007.

\bibitem{Marranghello}
G. F. Marranghello.
\\ Phase Transition In Rotating Neutron Stars
\\ International Journal of Modern Physics D 2007 16:02n03, 333-339 

\bibitem{1106.2736}
G. F. Burgio, V. Ferrari, L. Gualtieri and H. J. Schulze.
\\ Oscillations of hot, young neutron stars: Gravitational wave frequencies and damping times, 2011,
\\ Phys.Rev.D84:044017,2011;
\\ arXiv:1106.2736.
\\ DOI: 10.1103/PhysRevD.84.044017.

\bibitem{astro-ph/0412277}
Alessandra Buonanno, Guenter Sigl, Georg G. Raffelt, Hans-Thomas Janka and Ewald Mueller.
\\ Stochastic Gravitational Wave Background from Cosmological Supernovae, 2004,
\\ Phys.Rev. D72 (2005) 084001;
\\ arXiv:astro-ph/0412277.
\\ DOI: 10.1103/PhysRevD.72.084001.

\bibitem{hep-th/9211021}
M. Gasperini and G. Veneziano.
\\ Pre-Big-Bang in String Cosmology, 1992,
\\ Astropart.Phys.1:317-339,1993;
\\ arXiv:hep-th/9211021.
\\ DOI: 10.1016/0927-6505(93)90017-8.

\bibitem{hep-th/9701146}
A. Buonanno, M. Gasperini, M. Maggiore and C. Ungarelli.
\\ Expanding and contracting universes in third quantized string cosmology, 1997,
\\ Class.Quant.Grav.14:L97-L103,1997;
\\ arXiv:hep-th/9701146.
\\ DOI: 10.1088/0264-9381/14/5/005.

\bibitem{1006.0217}
Jean-Francois Dufaux, Daniel G. Figueroa and Juan Garcia-Bellido.
\\ Gravitational Waves from Abelian Gauge Fields and Cosmic Strings at Preheating, 2010,
\\ Phys.Rev.D82:083518,2010;
\\ arXiv:1006.0217.
\\ DOI: 10.1103/PhysRevD.82.083518.

\bibitem{Grishchuk1}
L.P. Grishchuk.
\\ Amplification of gravitational waves in an isotropic universe
\\ JETP, 1975, Vol. 40, No. 3, p. 409
\\ \url{http://www.jetp.ac.ru/cgi-bin/dn/e_040_03_0409.pdf}

\bibitem{Grishchuk2}
L. P. Grishchuk.
\\ Relic gravitational waves and limits on inflation
\\ Phys. Rev. D 48, 3513 – Published 15 October 1993
\\ \url{http://journals.aps.org/prd/pdf/10.1103/PhysRevD.48.3513}

\bibitem{0711.2593}
Chiara Caprini, Ruth Durrer and Geraldine Servant.
\\ Gravitational wave generation from bubble collisions in first-order phase transitions: an analytic approach, 2007,
\\ Phys.Rev.D77:124015,2008;
\\ arXiv:0711.2593.
\\ DOI: 10.1103/PhysRevD.77.124015.

\bibitem{0909.0622}
Chiara Caprini, Ruth Durrer and Geraldine Servant.
\\ The stochastic gravitational wave background from turbulence and magnetic fields generated by a first-order phase transition, 2009,
\\ JCAP 0912:024,2009;
\\ arXiv:0909.0622.
\\ DOI: 10.1088/1475-7516/2009/12/024.

\bibitem{0901.1661}
Chiara Caprini, Ruth Durrer, Thomas Konstandin and Geraldine Servant.
\\ General Properties of the Gravitational Wave Spectrum from Phase Transitions, 2009,
\\ Phys.Rev.D79:083519,2009;
\\ arXiv:0901.1661.
\\ DOI: 10.1103/PhysRevD.79.083519.

\bibitem{hep-th/0410222}
Thibault Damour and Alexander Vilenkin.
\\ Gravitational radiation from cosmic (super)strings: bursts, stochastic background, and observational windows, 2004,
\\ Phys.Rev. D71 (2005) 063510;
\\ arXiv:hep-th/0410222.
\\ DOI: 10.1103/PhysRevD.71.063510.

\bibitem{1106.5795}
Pablo A. Rosado.
\\ Gravitational wave background from binary systems, 2011,
\\ Physical Review D 84, 084004 (2011);
\\ arXiv:1106.5795.
\\ DOI: 10.1103/PhysRevD.84.084004.

\bibitem{astro-ph/0507028}
J. C. N. de Araujo and O. D. Miranda.
\\ Star formation rate density and the stochastic background of gravitational waves, 2005,
\\ Phys.Rev.D71:127503,2005;
\\ arXiv:astro-ph/0507028.
\\ DOI: 10.1103/PhysRevD.71.127503.


\end{thebibliography}
\end{document}































