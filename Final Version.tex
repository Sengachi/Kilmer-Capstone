\documentclass[twocolumn,11pt]{article}
\setlength{\textheight}{9truein}
\setlength{\topmargin}{-0.9truein}
\setlength{\parindent}{0pt}
\setlength{\parskip}{10pt}
\setlength{\columnsep}{.4in}
%
\newcommand{\beq}{\begin{equation}}
\newcommand{\eeq}{\end{equation}}
%
\usepackage{graphicx}
\usepackage{epsfig}
\usepackage{url}
\usepackage{cuted}
\usepackage{appendix}
\usepackage{bm}
%
\title{Removing the Astrophysical Gravitational Wave Foreground \\ \vspace{.1in} \large Comparing the Efficacies of Signal Deletion and Signal Subtraction with Respect to Gravitational Wave Telescope Parameters}
\author{Thomas Kilmer}
\date{18 Nov 2016}
%
%
\begin{document}
 \twocolumn[
   \begin{@twocolumnfalse}
   \maketitle
   \setlength{\parindent}{0pt}
   \begin{abstract} 

Gravitational waves from astrophysical sources have been detected and the next generation of gravitational wave telescopes might be able to detect gravitational waves from cosmic background sources. However we expect that these cosmic background signals will be obscured by a substantial astrophysical background which will make analyzing the cosmic background prohibitively difficult. In this paper we assess techniques for removing resolved astrophysical sources from gravitational wave data for the purpose of reducing the astrophysical background and easing analysis of the cosmic gravitational wave background. 
   
  \vspace{.3in} 
     \end{abstract}
    \end{@twocolumnfalse}]

\section*{Introduction}

Advanced LIGO recently detected gravitational waves from a binary black hole merger~\cite{1602.03837}. While estimates for the number of coalescing binary black holes in the universe are imprecise, there are many such binaries~\cite{0912.1074}. This large number of binary black holes produces overlapping, incoherent signals which form a ``background'' of gravitational waves~\cite{1604.02513}. This background is statistically homogeneous and isotropic, despite the discrete nature of the sources which make it up~\cite{1604.02513}. With current advances in gravitational wave detection this background will be measured within the next few years~\cite{1604.02513}.

!!! Insert description of background here. !!!

Binary black holes are not the only thing which contribute to the gravitational wave background. A number of other astrophysical sources contribute to the background including rotating neutron stars~\cite{1206.1330}, neutron star phase transitions~\cite{Marranghello}, instabilities in young neutron stars~\cite{1106.2736}, and core collapse supernovae~\cite{astro-ph/0412277}. There are also cosmological sources which contribute to the gravitational wave background including pre-Big Bang cosmological factors~\cite{hep-th/9211021}~\cite{hep-th/9701146}~\cite{1006.0217}, the amplification of weak gravitational waves~(including vacuum fluctuations) during inflation~\cite{Grishchuk1}~\cite{Grishchuk2}, phase transitions in the early universe~\cite{0711.2593}~\cite{0909.0622}~\cite{0901.1661}, and cosmic strings~\cite{hep-th/0410222}. The analysis of gravitational waves produced by the aforementioned cosmological sources could provide insight into the early processes of the universe.

However the analysis of the gravitational wave background produced by cosmological sources is complicated by the inability to separate it from the gravitational wave background produced by astrophysical sources~\cite{1604.02513}. However the astrophysical background can be separated into two parts, an unresolvable astrophysical background in which every source generates signals with signal to noise ratios less than 1 and a resolvable astrophysical background in which every source generates signals with SNRs greater than or equal to 1 called the astrophysical foreground~\cite{1106.5795}. The signals generated by the astrophysical foreground can then be identified and removed from the data, reducing the total astrophysical background and making it easier to identify any cosmic background which may exist. 

This project is not concerned with how the astrophysical foreground signals are identified, which has been discussed in great detail elsewhere~\cite{1602.03839}. The goal of this project is to develop tools for determining the efficacy of signal deletion and signal subtraction as methods to remove resolved astrophysical signals and to use these tools to determine which method will be most effective for the next generation of gravitational wave telescopes. In Capstone I, as preparation for this, we analyze a simple model that assumes flawless signal signal removal.





\section*{Resolvable Energy Fraction}

When removing astrophysical foreground signals, the point of the exercise is to reduce the energy of astrophysical signals recorded by a detector so that the astrophysical to cosmic signal energy ratio is as small as possible. After all, the less gravitational energy from astrophysical sources is recorded by the detector compared to cosmic sources, the easier it will be to accurately identify cosmic sources of gravitational energy. 

This makes energy a sensible quantity with which to quantify total and resolved astrophysical signals. It also lends itself to a simple metric for quantifying the efficacy of signal removal techniques, we want the fraction of resolved energy over total energy to be as close to 1 as possible. That is we want the dimensionless quantity $\alpha$ to be as close to 1 as possible, as $\alpha$ is shown in Equation~\ref{eq:alpha}.

\beq
\alpha = \frac{E_{res}}{E_{tot}}
\label{eq:alpha}
\eeq

To do so we need to determine the total astrophysical gravitational wave energy incident on an arbitrary detector~($E_{tot}$) and the resolvable energy incident on an arbitrary detector~($E_{res}$). 

For the purposes of this project we will be assuming that only astrophysical signals our detectors pick up come from black holes binaries, which is a reasonable approximation as black hole binaries are expected to constitute the vast majority of all astrophysical gravitational wave sources~\cite{1206.1330}.



\subsection*{Astrophysical Signal Density}



!!! Simplify following paragraph !!!

To describe the energy~($E$) recorded by gravitational wave telescopes we use the integral of gravitational wave signal density~($\Omega_{gw}(f)$) with respect to frequency~($f$), shown in Equation~\ref{eq:energy}. Signal density is a dimensionless quantity defined as the the energy density with respect to frequency~($f\frac{\partial \rho_{gw}(f)}{\partial f}$) of recorded gravitational waves as a proportion of the energy density needed to close the universe~($\rho_c$), as shown in Equation~\ref{eq:signal_density_compact}.~\cite{1112.1898}

\begin{eqnarray}
E &=& \int_{0}^{\infty} \Omega_{gw}(f)~df
\label{eq:energy} \\
\Omega_{gw}(f) &=& \frac{f~\frac{\partial \rho_{gw}(f)}{\partial f}}{\rho_c}
\label{eq:signal_density_compact} \\
\nonumber
\end{eqnarray}

Equation~\ref{eq:signal_density_compact} is not a particularly workable form for signal density though. Before we continue further we want to find a more accessible way of describing $\Omega_{gw}(f)$. 

Closure energy density~($\rho_c$) can be described with the Hubble constant~($H_0$), speed of light constant~($c$), and gravitational force constant~($G$), shown in Equation~\ref{eq:closure_energy}.~\cite{gravitation}

\beq
\rho_c = \frac{3 H_0 c^2}{8 \pi G}
\label{eq:closure_energy}
\eeq

The astrophysical energy density with respect to frequency~($f\frac{\partial \rho_{gw}(f)}{\partial f}$) can be divided by $c$ to change area per volume into area per volume, which replaces the energy density with integrated flux~($\int\Phi(f,M_c)dz$) as shown in Equation~\ref{eq:flux}. !!!!!!!!! Note that flux is dependent on chirp mass~($M_c$). Chirp mass~($M_c$) is a useful term for describing a binary's characteristics defined in terms of the masses of the binary's black holes~($M_1$ \& $M_2$) as shown in Equation~\ref{eq:chirp_mass}~\cite{1602.03837}.

\begin{eqnarray}
f\frac{\partial \rho_{gw}(f)}{\partial f} = \frac{1}{c}\int\Phi(f,M_c)dz
\label{eq:flux} \\
M_c = \frac{(M_1 M_2)^{3/5}}{(M_1 + M_2)^{1/5}}
\label{eq:chirp_mass} \\
\nonumber
\end{eqnarray}

Note that the transformation we make in Equation~\ref{eq:flux} introduces a chirp mass dependence into our description of signal density. This means any calculation of $E$ that we make will have a chirp mass dependence, $E$ will describe the gravitational wave energy of black hole binaries only of a given chirp mass rather than of all of them. This is dealt with later in the paper. 

From here we can expand upon flux~($\int\Phi(f,M_c)dz$). Flux is a product of the energy density produced by a single source~($\frac{\partial E_{source}(f,M_c)}{\partial f}$), the apparent distribution of gravitational wave sources with respect to redshift~($R_z(z)$), and the inverse surface area of the sphere over which the gravitational waves are spread~($\frac{1}{4 \pi ((1+z)r(z))^2}$), all of which is shown in Equation~\ref{eq:flux_expanded}. Note that the radius of the sphere over which gravitational waves are spread has radius~$r(z)$, where~$r(z)$ is the proper distance between the detector and the source. 

{\footnotesize
\beq
\Phi(f,M_c) = \frac{\partial E_{source}(f,M_c)}{\partial f}R_z(z)\frac{1}{4 \pi (1+z) r(z)}
\label{eq:flux_expanded}
\eeq}

The energy density produced by a single source~($\frac{\partial E_{source}(f,M_c)}{\partial f}$) is described in~\cite{1602.03847}, as shown in Equation~\ref{eq:source_energy}. Unfortunately this equation for energy density assumes a Newtonian orbit, which means any signal density equation we define using this energy density will be restricted to low signal frequencies where the black holes are orbiting one another at large distances and it is reasonable to make a Newtonian approximation of their orbit. 

{\footnotesize
\beq
\frac{\partial E_{source}(f,M_c)}{\partial f} = \frac{(G \pi)^{2/3}}{3}((1+z)M_c)^{5/3}f^{-1/3}
\label{eq:source_energy}
\eeq}

The apparent distribution of gravitational wave sources with respect to redshift~($R_z(z)$) is a product of the actual distribution of gravitational wave sources~($R_V(z)$), the expansion of space-time with respect to redshift~($\frac{\partial V(z)}{\partial z}$), and a constant fudge factor $\lambda$ which can accounts for the observed incidence of black hole binaries, all of which is shown in Equation~\ref{eq:Rz}.

\beq
R_z(z) = \lambda R_V(z) \frac{\partial V(z)}{\partial z}
\label{eq:Rz}
\eeq

The actual distribution of gravitational wave source~$R_V(z)$ is not well determined, for the purposes of this paper we will use the function for $R_V(z)$ described in~\cite{1602.04531}, shown in Equation~\ref{eq:Rv}.

\beq
R_V(z) = 0.015~\frac{(1+z)^{2.7}}{1 + [\frac{1+z}{2.9}]^{5.6}}
\label{eq:Rv}
\eeq

The expansion of space-time with respect to redshift~($\frac{\partial V(z)}{\partial z}$) can be described as the proper surface area of the sphere over which the gravitational waves are spread~($4 \pi r(z)^2$) divided by the Hubble Constant in units of distance~($\frac{H_0}{c}$) and a factor which describes the the dependence of the volume of spacetime surrounding an observed source with respect to redshift~($E(\Omega_M,\Omega_\Lambda,z)$), as shown in Equation~\ref{eq:volume}.

\beq
\frac{\partial V(z)}{\partial z} = \frac{4 \pi r(z)^2 c}{H_0 E(\Omega_M,\Omega_\Lambda,z)}
\label{eq:volume}
\eeq

For $E(\Omega_M,\Omega_\Lambda,z)$ we use the function that is described in~\cite{1602.04531}, which is well agreed upon. This is shown in Equation~\ref{eq:comoving}.

\beq
E(\Omega_M,\Omega_\Lambda,z) = \sqrt{\Omega_M (1+z)^3 + \Omega_{\Lambda}}
\label{eq:comoving}
\eeq

Now we have finally split $\Omega_{gw}(f,M_c)$ into a slew of parts which we can numerically evaluate (for low frequencies at least). The last thing we need to do to get a numerical description of $\Omega_{gw}(f,M_c)$ is to describe the values of the constants embedded in those parts. 

The speed of light~($c$) is $3*10^8\frac{m}{s}$. The gravitational force constant~($G$) is $6.674*10^{-11}\frac{m^3}{kg~s^2}$. Using the standard $\Lambda$CDM cosmology~\cite{1502.01589} we can define the Hubble constant~($H_0$) as $70\frac{km}{s~Mpc}$, $\Omega_M$ as 0.3, and $\Omega_\Lambda$ as 0.7. Lastly we need to define $\lambda$. Unfortunately not enough black hole binaries have been detected yet to accurately determine $\lambda$, so we are forced to simply make a best guess. !!!

Pulling together Equations~\ref{eq:signal_density_compact}, \ref{eq:closure_energy}, \ref{eq:flux}, \ref{eq:flux_expanded}, \ref{eq:source_energy}, \ref{eq:Rz}, \ref{eq:Rv}, \ref{eq:volume}, and \ref{eq:comoving} we can construct Equation~\ref{eq:signal_density_low} as an equation with which to describe signal density for low frequencies. 

{\tiny
\beq
\Omega_{gw}^{low-f}(f,M_c) = \frac{8\lambda(\pi G)^{5/3}}{9 H_0^3 c^2} M_c^{5/3} f^{2/3}\int_{0}^{z_{max}} \frac{R_V(z)~dz}{(1+z)^{1/3} E[\Omega_M, \Omega_{\Lambda}, z]}
\label{eq:signal_density_low} 
\eeq}

This is a reasonable low-frequency approximation for gravitational wave signal density, but to turn signal density into an energy quantity we need to integrate signal density from 0~Hz to $\infty$~Hz so a low-frequency approximation just isn't enough.~(Especially since $\Omega_{gw}^{low-f}(f,M_c)$ would integrate to infinity under such bounds). 

To define signal density in a way that is valid for all frequencies, let us first ask what happens to black hole binaries at high frequency. 

As black hole binary systems radiate energy in the form of gravitational waves the black holes spiral inward towards one another and the frequency with which they orbit one another increases. However at some point the black holes will be so close that their Schwarzschild radii intersect and the two black holes merge into one. Once the black hole binary has merged into a single black hole it ceases to emit gravitational wave energy of a magnitude that gravitational wave telescopes can detect. So black hole binaries will not radiate energy beyond a particular cutoff frequency, the highest orbital frequency the black holes reach before merging. 

Furthermore at higher frequencies where the black holes are very close to one another, relativistic effects begin to dominate and our Newtonian approximation of flux~($\Phi(f,M_c)$) does not adequately match reality. 

As it so happens detailed models of binary strain~($h(f,M_c)$) exist~\cite{0909.2867} which show an appropriate cutoff at a given binary's merger frequency and which take general relativity into account. Furthermore strain can be related to flux~($\Phi(f)$) with respect to frequency~($f$), which gives us a way to use these models of binary strain to describe the behavior of signal density for all frequencies. 

From~\cite{gravitation} we can retrieve the proportionalities in Equation~\ref{eq:proportions}, which taken together show the proportionality of flux to strain.

\beq
\Phi(f,M_c) \propto \frac{\partial E}{\partial f} \propto f^2~|h(f,M_c)|^2
\label{eq:proportions}
\eeq 

This does not fully describe the frequency dependence of the proportionality between flux and strain though. As shown above in !!!!!! flux has a frequency dependence of $f^{-1/3}$ (for a more intuitive explanation of why this is see Appendix~\ref{ap:flux}). 

Therefore the proportionality between strain and and the non-frequency components of flux is as shown in Equation~\ref{eq:proportions2}. 

\beq
\Phi(M_c) \propto f^{7/3}~|h(f,M_c)|^2
\label{eq:proportions2}
\eeq 

Because $\Phi(M_c)$ is part of our low-frequency approximation of signal density and not dependent on frequency, we can say that $\Omega_{gw}^{low-f}(f,M_c)$ is proportional without respect to frequency to the non-frequency components of flux. This allows us to describe a proportionality between signal density and strain, shown in Equation~\ref{eq:proportions3}.

\beq
\Omega_{gw}(f,M_c) \propto \Phi(M_c) \propto f^{7/3}~|h(f,M_c)|^2
\label{eq:proportions3}
\eeq 

From this we can immediately create another proportionality, shown in Equation~\ref{eq:proportions4}.

{\scriptsize
\beq
\frac{\Omega_{gw}(f,M_c)}{\Omega_{gw}^{low-f}(10Hz,M_c)} \propto \frac{f^{7/3}~|h(f,M_c)|^2}{(10Hz)^{7/3}~|h(10Hz,M_c)|^2}
\label{eq:proportions4}
\eeq}

At 10~Hz Newtonian effects dominate the inspiral of binary black holes and $\Omega_{gw}^{low-f}(10Hz,M_c)$ returns accurate values. So by solving the proportionality in Equation~\ref{eq:proportions4} using our equation for $\Omega_{gw}^{low-f}(10Hz,M_c)$ and using the values for $h(f)$ given in~\cite{0909.2867} we can calculate high frequency values for signal density, as shown in Equation~\ref{eq:signal_density}.

{\tiny
\beq
\Omega_{gw}(f,M_c) = (\frac{f}{10Hz})^{7/3}\frac{|h(f,M_c)|^2}{|h(10Hz,M_c)|^2}\Omega_{gw}^{low-f}(f,M_c)
\label{eq:signal_density}
\eeq}

This equation accurately describes the gravitational wave signal density for black hole binaries for all frequency values, though as mentioned earlier it still has an unwanted chirp mass dependence that needs to be dealt with before it can be integrated with respect to frequency to find an energy quantity. 





\subsection*{Total Astrophysical Signal Energy}






\subsection*{Resolvable Astrophysical Signal Energy}





\clearpage

\section*{Appendix}

\subsection{Flux Frequency Dependence}
\label{ap:flux}

As described in Equation~\ref{eq:proportions}, $\Phi(f,M_c) \propto \frac{\partial E}{\partial f}$. Because our $\Phi(f,M_c)$ operates off the assumption of a Newtonian orbit, we can choose to find $\frac{\partial E}{\partial f}$ using Kepler's laws of orbital mechanics, shown in Equation~\ref{eq:kepler1} and Equation~\ref{eq:kepler2}. 

\begin{eqnarray}
E &=& -\frac{M^2}{2a} 
\label{eq:kepler1} \\
2 \pi f &=& (\frac{M}{a^3})^{1/2}
\label{eq:kepler2} \\
\nonumber
\end{eqnarray}

In Equations~\ref{eq:kepler1} and \ref{eq:kepler2} $M$ describes the masses of each orbiting body, assuming the orbiting bodies have identical mass. Which is a reasonable enough approximation to make for this qualitative description of flux's frequency dependence. Also, $a$ is the centripetal acceleration of the orbiting bodies, $E$ is the gravitational potential energy contained within their orbit, and $f$ is their orbital frequency (which is the same thing as their signal frequency). 

From Equations~\ref{eq:kepler1} and \ref{eq:kepler2} we can derive Equation~\ref{eq:energy} and take its derivative with respect to frequency, $f$, to find Equation~\ref{eq:flux_newtonian}. 

\begin{eqnarray}
E &=& -\frac{M}{2}(M \pi f)^{2/3}
\label{eq:energy} \\ 
\frac{\partial E}{\partial f} &=& \frac{M^2 \pi}{3}(M \pi f)^{-1/3}
\label{eq:flux_newtonian}
\end{eqnarray}

Therefore we can say that $\frac{\partial E}{\partial f} \propto f^{-1/3}$ and from there say that $\Phi(f,M_c) \propto f^{-1/3}$. This is hopefully a more satisfactory explanation of why flux has a frequency dependence of $f^{-1/3}$ than simply citing some equation pulled from the aether. 








\begin{thebibliography}{100}

	\bibitem{1602.03837} The LIGO Scientific Collaboration and  the Virgo Collaboration.
\\ Observation of Gravitational Waves from a Binary Black Hole Merger, 2016,
\\ Phys. Rev. Lett. 116, 061102~(2016);
\\ arXiv:1602.03837.
\\ DOI: 10.1103/PhysRevLett.116.061102.

	\bibitem{0912.1074} Ilya Mandel and Richard O'Shaughnessy.
\\ Compact Binary Coalescences in the Band of Ground-based Gravitational-Wave Detectors, 2009,
\\ Class.Quant.Grav.27:114007,2010;
\\ arXiv:0912.1074.
\\ DOI: 10.1088/0264-9381/27/11/114007.

	\bibitem{1604.02513} Thomas Callister, Letizia Sammut, Eric Thrane, Shi Qiu and Ilya Mandel.
\\ The limits of astrophysics with gravitational wave backgrounds, 2016;
\\ arXiv:1604.02513.

	\bibitem{1206.1330} Pablo A. Rosado.
\\ Gravitational wave background from rotating neutron stars, 2012,
\\ Physical Review D 86, 104007~(2012);
\\ arXiv:1206.1330.
\\ DOI: 10.1103/PhysRevD.86.104007.

	\bibitem{Marranghello} G. F. Marranghello.
\\ Phase Transition In Rotating Neutron Stars
\\ International Journal of Modern Physics D 2007 16:02n03, 333-339

	\bibitem{1106.2736} G. F. Burgio, V. Ferrari, L. Gualtieri and H. J. Schulze.
\\ Oscillations of hot, young neutron stars: Gravitational wave frequencies and damping times, 2011,
\\ Phys.Rev.D84:044017,2011;
\\ arXiv:1106.2736.
\\ DOI: 10.1103/PhysRevD.84.044017.

	\bibitem{astro-ph/0412277} Alessandra Buonanno, Guenter Sigl, Georg G. Raffelt, Hans-Thomas Janka and Ewald Mueller.
\\ Stochastic Gravitational Wave Background from Cosmological Supernovae, 2004,
\\ Phys.Rev. D72~(2005) 084001;
\\ arXiv:astro-ph/0412277.
\\ DOI: 10.1103/PhysRevD.72.084001.

	\bibitem{hep-th/9211021} M. Gasperini and G. Veneziano.
\\ Pre-Big-Bang in String Cosmology, 1992,
\\ Astropart.Phys.1:317-339,1993;
\\ arXiv:hep-th/9211021.
\\ DOI: 10.1016/0927-6505(93)90017-8.

	\bibitem{hep-th/9701146} A. Buonanno, M. Gasperini, M. Maggiore and C. Ungarelli.
\\ Expanding and contracting universes in third quantized string cosmology, 1997,
\\ Class.Quant.Grav.14:L97-L103,1997;
\\ arXiv:hep-th/9701146.
\\ DOI: 10.1088/0264-9381/14/5/005.

	\bibitem{1006.0217} Jean-Francois Dufaux, Daniel G. Figueroa and Juan Garcia-Bellido.
\\ Gravitational Waves from Abelian Gauge Fields and Cosmic Strings at Preheating, 2010,
\\ Phys.Rev.D82:083518,2010;
\\ arXiv:1006.0217.
\\ DOI: 10.1103/PhysRevD.82.083518.

	\bibitem{Grishchuk1} L.P. Grishchuk.
\\ Amplification of gravitational waves in an isotropic universe
\\ JETP, 1975, Vol. 40, No. 3, p. 409
\\ \url{http://www.jetp.ac.ru/cgi-bin/dn/e_040_03_0409.pdf}

	\bibitem{Grishchuk2} L. P. Grishchuk.
\\ Relic gravitational waves and limits on inflation
\\ Phys. Rev. D 48, 3513 – Published 15 October 1993
\\ \url{http://journals.aps.org/prd/pdf/10.1103/PhysRevD.48.3513}

	\bibitem{0711.2593} Chiara Caprini, Ruth Durrer and Geraldine Servant.
\\ Gravitational wave generation from bubble collisions in first-order phase transitions: an analytic approach, 2007,
\\ Phys.Rev.D77:124015,2008;
\\ arXiv:0711.2593.
\\ DOI: 10.1103/PhysRevD.77.124015.

	\bibitem{0909.0622} Chiara Caprini, Ruth Durrer and Geraldine Servant.
\\ The stochastic gravitational wave background from turbulence and magnetic fields generated by a first-order phase transition, 2009,
\\ JCAP 0912:024,2009;
\\ arXiv:0909.0622.
\\ DOI: 10.1088/1475-7516/2009/12/024.

	\bibitem{0901.1661} Chiara Caprini, Ruth Durrer, Thomas Konstandin and Geraldine Servant.
\\ General Properties of the Gravitational Wave Spectrum from Phase Transitions, 2009,
\\ Phys.Rev.D79:083519,2009;
\\ arXiv:0901.1661.
\\ DOI: 10.1103/PhysRevD.79.083519.

	\bibitem{hep-th/0410222} Thibault Damour and Alexander Vilenkin.
\\ Gravitational radiation from cosmic~(super)strings: bursts, stochastic background, and observational windows, 2004,
\\ Phys.Rev. D71~(2005) 063510;
\\ arXiv:hep-th/0410222.
\\ DOI: 10.1103/PhysRevD.71.063510.

	\bibitem{1106.5795} Pablo A. Rosado.
\\ Gravitational wave background from binary systems, 2011,
\\ Physical Review D 84, 084004~(2011);
\\ arXiv:1106.5795.
\\ DOI: 10.1103/PhysRevD.84.084004.

	\bibitem{1602.03839} The LIGO Scientific Collaboration,  the Virgo Collaboration
\\ GW150914: First results from the search for binary black hole coalescence with Advanced LIGO, 2016,
\\ Phys. Rev. D 93, 122003~(2016);
\\ arXiv:1602.03839.
\\ DOI: 10.1103/PhysRevD.93.122003.

	\bibitem{1112.1898} Chengjian Wu, Vuk Mandic and Tania Regimbau.
\\ Accessibility of the Gravitational-Wave Background due to Binary Coalescences to Second and Third Generation Gravitational-Wave Detectors, 2011;
\\ arXiv:1112.1898.
\\ DOI: 10.1103/PhysRevD.85.104024.

	\bibitem{1602.03847} The LIGO Scientific Collaboration and  the Virgo Collaboration.
\\ GW150914: Implications for the stochastic gravitational wave background from binary black holes, 2016,
\\ Phys. Rev. Lett. 116, 131102~(2016);
\\ arXiv:1602.03847.
\\ DOI: 10.1103/PhysRevLett.116.131102.

	\bibitem{1602.04531} Krzysztof Belczynski, Daniel E. Holz, Tomasz Bulik and Richard O'Shaughnessy.
\\ The first gravitational-wave source from the isolated evolution of two 40-100 Msun stars, 2016;
\\ arXiv:1602.04531.
\\ DOI: 10.1038/nature18322.

	\bibitem{1405.7016} M. Dominik, E. Berti, R. O'Shaughnessy, I. Mandel, K. Belczynski, C. Fryer, D. Holz, T. Bulik and F. Pannarale.
\\ Double Compact Objects III: Gravitational Wave Detection Rates, 2014;
\\ arXiv:1405.7016.
\\ DOI: 10.1088/0004-637X/806/2/263.

	\bibitem{math.9909058} Ivan Mirkovic and Dmitriy Rumynin.
\\ Geometric Representation Theory of Restricted Lie Algebras of Classical Type, 1999;
\\ arXiv:math/9909058.

	\bibitem{1012.0908} S Hild, et al. 
\\ Sensitivity Studies for Third-Generation Gravitational Wave Observatories, 2010;
\\ arXiv:1012.0908.
\\ DOI: 10.1088/0264-9381/28/9/094013.

	\bibitem{1607.08697} LIGO Scientific Collaboration
\\ Exploring the Sensitivity of Next Generation Gravitational Wave Detectors, 2016;
\\ arXiv:1607.08697.

	\bibitem{gravitation} Misner, C.W. and Thorne, K.S. and Wheeler, J.A.
\\ Gravitation, 1973
\\ ISBN:9780716703440.
\\ lCCN:78156043.

	\bibitem{0909.2867} P. Ajith, M. Hannam, S. Husa, Y. Chen, B. Bruegmann, N. Dorband, D. Mueller, F. Ohme, D. Pollney, C. Reisswig, L. Santamaria and J. Seiler.
\\ Inspiral-merger-ringdown waveforms for black-hole binaries with non-precessing spins, 2009,
\\ Phys.Rev.Lett.106:241101,2011;
\\ arXiv:0909.2867.
\\ DOI: 10.1103/PhysRevLett.106.241101.

	\bibitem{astro-ph/0507028} J. C. N. de Araujo and O. D. Miranda.
\\ Star formation rate density and the stochastic background of gravitational waves, 2005,
\\ Phys.Rev.D71:127503,2005;
\\ arXiv:astro-ph/0507028.
\\ DOI: 10.1103/PhysRevD.71.127503.

\bibitem{1502.01589}
Planck Collaboration
\\ Planck 2015 results. XIII. Cosmological parameters, 2015,
\\ A\&A 594, A13 (2016);
\\ arXiv:1502.01589.
\\ DOI: 10.1051/0004-6361/201525830.


\end{thebibliography}
\end{document}








